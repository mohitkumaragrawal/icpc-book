\documentclass[a4paper, twocolumn]{article}
\usepackage[margin=1in]{geometry}
\usepackage[Glenn]{fncychap}
\usepackage{listings}
\usepackage{color}
\usepackage{verbatim}
\usepackage{geometry}
\usepackage{fancyhdr}

\title{TEAM DEAD TEMPLATES}
\author{mhtkrag}

\definecolor{dkgreen}{rgb}{0,0.6,0}
\definecolor{gray}{rgb}{0.5,0.5,0.5}
\definecolor{mauve}{rgb}{0.58,0,0.82}

\setlength\columnseprule{0.4pt}
\geometry{a4paper,left=1cm,right=1cm,top=2cm,bottom=1.5cm}

\lstset{frame=tb,
  language=c++,
  aboveskip=3mm,
  belowskip=3mm,
  showstringspaces=false,
  columns=flexible,
  basicstyle={\small\ttfamily},
  numbers=none,
  numberstyle=\tiny\color{gray},
  keywordstyle=\color{blue},
  commentstyle=\color{dkgreen},
  stringstyle=\color{mauve},
  breaklines=true,
  breakatwhitespace=true
  tabsize=2
}

\begin{document}
\begin{titlepage}
\maketitle
\thispagestyle{empty}
\pagebreak
%\pagestyle{plain}
\pagestyle{fancy}
\lhead{}
\rhead{}
\chead{TEAM DEAD}
\cfoot{}
\tableofcontents
\end{titlepage}

\pagestyle{fancy}
\chead{TEAM DEAD}
\cfoot{- \thepage \ -}
  
\section{data-structures}
\subsection{dsu}
\lstinputlisting{codes/data-structures/dsu.cpp}
\subsection{fenwick-tree}
\lstinputlisting{codes/data-structures/fenwick-tree.cpp}
\subsection{lazy-segment-tree}
\lstinputlisting{codes/data-structures/lazy-segment-tree.cpp}
\subsection{segment-tree-beats}
\lstinputlisting{codes/data-structures/segment-tree-beats.cpp}
\subsection{segment-tree}
\lstinputlisting{codes/data-structures/segment-tree.cpp}
\subsection{sparse-table}
\lstinputlisting{codes/data-structures/sparse-table.cpp}
\section{dp}
\subsection{sos-dp}
\lstinputlisting{codes/dp/sos-dp.cpp}
\section{graph}
\subsection{bellaman-ford}
\lstinputlisting{codes/graph/bellaman-ford.cpp}
\subsection{floyd-warshall}
\lstinputlisting{codes/graph/floyd-warshall.cpp}
\subsection{scc}
\lstinputlisting{codes/graph/scc.cpp}
\section{misc}
\subsection{bitwise-tricks}
\lstinputlisting{codes/misc/bitwise-tricks.cpp}
\subsection{cppt}
\lstinputlisting{codes/misc/cppt.cpp}
\subsection{ordered-set}
\lstinputlisting{codes/misc/ordered-set.cpp}
\section{number-theory}
\subsection{crt}
\lstinputlisting{codes/number-theory/crt.cpp}
\subsection{euler-totient-function}
\lstinputlisting{codes/number-theory/euler-totient-function.cpp}
\subsection{extended-euclid}
\lstinputlisting{codes/number-theory/extended-euclid.cpp}
\subsection{modular-int}
\lstinputlisting{codes/number-theory/modular-int.cpp}
\subsection{polard-rho}
\lstinputlisting{codes/number-theory/polard-rho.cpp}
\subsection{sieve}
\lstinputlisting{codes/number-theory/sieve.cpp}
\section{strings}
\subsection{kmp}
\lstinputlisting{codes/strings/kmp.cpp}
\subsection{z-algorithm}
\lstinputlisting{codes/strings/z-algorithm.cpp}
\section{tree}
\subsection{binary-lifting}
\lstinputlisting{codes/tree/binary-lifting.cpp}
\subsection{euler-tour}
\lstinputlisting{codes/tree/euler-tour.cpp}
\subsection{hld}
\lstinputlisting{codes/tree/hld.cpp}
\subsection{tree-isomorphism}
\lstinputlisting{codes/tree/tree-isomorphism.cpp}
\subsection{tree-lifing}
\lstinputlisting{codes/tree/tree-lifing.cpp}
\end{document}